% ====================================================================
% SYSTEM BLOCK DIAGRAM - RESEARCH GRADE (BLACK & WHITE + RED POWER)
% ====================================================================

\documentclass[border=15pt]{standalone}
\usepackage{tikz}
\usepackage{times} % Use standard Times font for research papers
\usetikzlibrary{shapes.geometric, arrows.meta, positioning, fit, calc, shadows}

% ==================== STYLES ====================

% Standard Functional Block (Hardware & Software)
\tikzstyle{block} = [
    rectangle,
    draw=black,
    line width=0.8pt,
    minimum width=2.8cm,
    minimum height=1.2cm,
    text centered,
    align=center, % Ensures multiline text breaks correctly
    font=\small\bfseries,
    fill=white,
    rounded corners=2pt,
    drop shadow={opacity=0.3, shadow xshift=2pt, shadow yshift=-2pt}
]

% Container Boxes (Subsystems)
\tikzstyle{container} = [
    rectangle,
    draw=black,
    dashed,
    line width=0.8pt,
    inner sep=15pt,
    rounded corners=5pt
]

% Lines
\tikzstyle{rfLine} = [
    {Stealth[length=3mm]}-{Stealth[length=3mm]},
    draw=black,
    dashed,
    line width=1pt
]

\tikzstyle{dataLine} = [
    -{Stealth[length=3mm]},
    draw=black,
    line width=1pt
]

\tikzstyle{powerLine} = [
    -{Stealth[length=3mm]},
    draw=red,       % ONLY Power lines are red
    line width=1pt
]

% Labels
\tikzstyle{linkLabel} = [
    font=\scriptsize,
    fill=white,
    inner sep=2pt,
    text=black
]

\begin{document}

\begin{tikzpicture}[node distance=2.0cm]

    % ==================== 1. AIR SEGMENT ====================
    \node[block] (airUnit) {Air Unit\\(nRF Sender)};

    % ==================== 2. GROUND HARDWARE SEGMENT ====================
    
    % nRF Receiver
    \node[block, below=2.5cm of airUnit] (nrf) {nRF Receiver\\Module};
    
    % Power Source (Top Right)
    \node[block, draw=red, text=red, dashed, right=1.8cm of nrf] (power) {USB Power Source\\(5V Input)};
    
    % ESP8266 Controller (Bottom Right)
    \node[block, below=1.0cm of power] (esp32) {ESP8266\\Controller};

    % Container for Ground HW
    \node[container, fit=(nrf) (esp32) (power)] (hwBox) {};
    % Label placed INSIDE (Bottom Left - Empty Space)
    \node[font=\bfseries, anchor=south west, xshift=10pt, yshift=5pt] at (hwBox.south west) {Ground Hardware Acquisition};

    % ==================== 3. GROUND SOFTWARE SEGMENT ====================
    
    % Web Serial API (Placed vertically below nRF)
    \node[block, below=5.5cm of nrf] (webSerial) {Web Serial\\Interface};
    
    % Data Parsing
    \node[block, right=1.8cm of webSerial] (parser) {JSON Data\\Parser};
    
    % Visualization Components
    \node[block, right=1.8cm of parser, yshift=1.4cm] (map) {Leaflet Map\\Engine};
    
    \node[block, right=1.8cm of parser] (gauges) {Telemetry\\Dashboard};
    
    \node[block, right=1.8cm of parser, yshift=-1.4cm] (logs) {System\\Data Logs};
    
    % Container for Software
    \node[container, fit=(webSerial) (parser) (map) (gauges) (logs)] (swBox) {};
    % Label placed INSIDE (Bottom Left - Empty Space below WebSerial)
    \node[font=\bfseries, anchor=south west, xshift=10pt, yshift=5pt] at (swBox.south west) {Ground Control Software (Visualization)};

    % ==================== CONNECTIONS ====================

    % Air -> Ground (RF)
    \draw[rfLine] (airUnit) -- node[linkLabel] {2.4GHz Wireless Telemetry} (nrf);

    % nRF -> ESP32 (SPI)
    % Route: Down from nRF then Right to ESP32 (Clean L-shape)
    \draw[dataLine] (nrf.south) |- node[linkLabel, near start] {SPI Bus} (esp32.west);

    % ESP32 -> Web Serial (USB)
    % Route: Down from ESP32, then Left across the gap, then Down to Web Serial
    % This avoids cutting through the JSON Parser block.
    \draw[dataLine] (esp32.south) -- ++(0,-0.8) -| node[linkLabel, near start, fill=white] {USB / UART Stream} (webSerial.north);

    % Power Routing (Red)
    % USB -> ESP8266 (Down - Input)
    \draw[powerLine] (power) -- node[linkLabel, text=red] {5V} (esp32);
    
    % ESP8266 -> nRF (Up then Left - Output via VIN)
    % Route: From Top-Left corner of ESP, go LEFT into the gap, then UP to nRF level, then LEFT to nRF.east
    \draw[powerLine] (esp32.north west) -- ++(-0.6, 0) |- node[linkLabel, text=red, near end] {5V (VIN)} (nrf.east);

    % Software Data Flow
    \draw[dataLine] (webSerial) -- node[linkLabel] {Raw Bytes} (parser);
    
    % Parser distribution (Branching)
    \draw[dataLine] (parser) -- (gauges);
    \draw[dataLine] (parser.east) -- ++(0.5,0) |- (map.west);
    \draw[dataLine] (parser.east) -- ++(0.5,0) |- (logs.west);

    % ==================== LEGEND ====================
    % Moved to bottom of Software Box
    \node[draw=black, line width=0.5pt, anchor=north west, fill=white, inner sep=5pt] at ([yshift=-0.5cm]swBox.south west) (legend) {
        \begin{tikzpicture}[baseline=(current bounding box.center)]
            \node[font=\bfseries\scriptsize, anchor=west] at (0,0.2) {\underline{Legend:}};
            \draw[dataLine] (0,-0.3) -- (1.0,-0.3) node[right, font=\tiny] {Signal Flow};
            \draw[rfLine] (0,-0.7) -- (1.0,-0.7) node[right, font=\tiny] {Wireless RF};
            \draw[powerLine] (0,-1.1) -- (1.0,-1.1) node[right, font=\tiny] {Power Line};
        \end{tikzpicture}
    };

\end{tikzpicture}

\end{document}
